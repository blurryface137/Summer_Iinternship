\documentclass{beamer}

\usepackage[utf8]{inputenc}
\usetheme{Madrid}
\usecolortheme{default}
\usepackage[T2A]{fontenc}
\usepackage[utf8]{inputenc}
%\usepackage[russian]{babel}
\usepackage{amsmath}

%~~~~~~~~~~~~~~~~~~~~~~~~~~~~~~~~~~~~~~~~~~~~~~~~~~~~~~~~~~~~~~~~~~~~~~~~~~~~~~~

\title[HSE Internship]{LaTeX Presentation}
\subtitle{\textit{or an attempt at this}}
\author[Nekrasov]{Artem Nekrasov}
\date[July 2022]{July 2022}

%~~~~~~~~~~~~~~~~~~~~~~~~~~~~~~~~~~~~~~~~~~~~~~~~~~~~~~~~~~~~~~~~~~~~~~~~~~~~~~~

\AtBeginSection[]
{
  \begin{frame}
    \frametitle{Table of Contents}
    \tableofcontents[currentsection]
  \end{frame}
}
%~~~~~~~~~~~~~~~~~~~~~~~~~~~~~~~~~~~~~~~~~~~~~~~~~~~~~~~~~~~~~~~~~~~~~~~~~~~~~~~


\begin{document}
\frame{\titlepage}

\begin{frame}
\frametitle{Table of Contents}
\tableofcontents
\end{frame}

%~~~~~~~~~~~~~~~~~~~~~~~~~~~~~~~~~~~~~~~~~~~~~~~~~~~~~~~~~~~~~~~~~~~~~~~~~~~~~~~

\section{A few words at the beginning}

%~~~~~~~~~~~~~~~~~~~~~~~~~~~~~~~~~~~~~~~~~~~~~~~~~~~~~~~~~~~~~~~~~~~~~~~~~~~~~~~
\begin{frame}
\frametitle{What is the power of LaTeX?}

\begin{itemize}
    \item The possibility of unlimited customization
    \item Free layout system for high-quality documents
    \item Ease of working with complex mathematical formulas
    \item The main task of the author is not design, but content
    \item Multi-lingual typesetting
\end{itemize}
\end{frame}

%~~~~~~~~~~~~~~~~~~~~~~~~~~~~~~~~~~~~~~~~~~~~~~~~~~~~~~~~~~~~~~~~~~~~~~~~~~~~~~~

\section{Markdown}

%~~~~~~~~~~~~~~~~~~~~~~~~~~~~~~~~~~~~~~~~~~~~~~~~~~~~~~~~~~~~~~~~~~~~~~~~~~~~~~~

\begin{frame}
\frametitle{Markdown}

\begin{itemize}
    \item Using the \textit{underbrace} command, you can draw a line under an important word;
        \\ $\underline{Example!}$
    \item Using the \textit{underscore} command, you can combine the information with a parenthesis at the bottom;
        \\$\underbrace{a + b + c + d}_{Example!} + e + f$
    \item Using the \textit{xrightarrow} command, you can get an arrow with an inscription on top;
        \\$Example \xrightarrow[]{function "!"} Example!$
    \item Using the \textit{star} command, you can get an asterisk-symbol;
        \\$\star Example!$
\end{itemize}
\end{frame}

%~~~~~~~~~~~~~~~~~~~~~~~~~~~~~~~~~~~~~~~~~~~~~~~~~~~~~~~~~~~~~~~~~~~~~~~~~~~~~~~

\section{A few words at the end}

%~~~~~~~~~~~~~~~~~~~~~~~~~~~~~~~~~~~~~~~~~~~~~~~~~~~~~~~~~~~~~~~~~~~~~~~~~~~~~~~


\begin{frame}
\frametitle{Impressions}

This mini-course allowed me to finally start learning LaTeX. The word "forced" is inappropriate here, because I got incredible pleasure from the opportunities that this system provides. After I finally "assembled" the final file, I realized what the real quality of the document is. It's like I wrote a book or a scientific article myself. It's inspiring.

\end{frame}

%~~~~~~~~~~~~~~~~~~~~~~~~~~~~~~~~~~~~~~~~~~~~~~~~~~~~~~~~~~~~~~~~~~~~~~~~~~~~~~~

\end{document}